\documentclass[12pt,oneside,a4paper]{article}

\usepackage[utf8]{inputenc} % Lærer LaTeX at forstå unicode - HUSK at filen skal
% være unicode (UTF-8), standard i Linux, ikke i
% Win.

\usepackage[danish]{babel} % Så der fx står Figur og ikke Figure, Resumé og ikke
% Abstract etc. (god at have).

%\usepackage{graphicx}
\usepackage{amsfonts}
\usepackage{amsthm}        % Theorems
\usepackage{amsmath}
%\usepackage{hyperref}

%\renewcommand{\mid}[1]{{\rm E}\!\left[#1\right]}
\newcommand{\bas}{\begin{eqnarray*}}
\newcommand{\eas}{\end{eqnarray*}}
\newcommand{\be}{\begin{equation}}
\newcommand{\ee}{\end{equation}}
\newcommand{\bea}{\begin{eqnarray}}
\newcommand{\eea}{\end{eqnarray}}

\newtheorem{thm}{Sætning}[section]
\newtheorem{mydef}[thm]{Definition}
\newtheorem{eks}[thm]{Eksempel}

\title{Meyer -- matematisk løsning}
\date{Juni 2016}
\author{Michael Jørgensen}

\begin{document}

\maketitle

Dette dokument præsenterer en matematisk optimal strategi for at spille spillet
"Meyer".

\section{Regler}

Først gennemgår jeg kort reglerne for spillet.  I spillet Meyer kaster man to
terninger (som man kun selv kan se).  Man kigger på terningerne, afgiver en
melding, og giver de skjulte terninger videre til den næste spiller.

Den næste spiller har nu to muligheder: Enten at "løfte" eller at "ryste". Man
"løfter" terningerne, hvis man ikke tror på meldingen. Så bliver terningerne
vist, og hvis de viser det samme som eller bedre end den sidste melding, så har
spilleren tabt. Hvis de viser mindre end den sidste melding, så har den sidste
spiller (som har afgivet meldingen) tabt.

Hvis en spiller "ryster", så bliver terningerne rystet, og spilleren kigger på
terningerne, og afgiver en ny melding. Denne nye melding skal være mindst lige
så god som den sidste melding.

Spillet fortsætter således med meldinger, som aldrig bliver dårligere, indtil en spiller vælger at løfte.

De to terninger har i alt $6+5+4+3+2+1=21$ forskellige visninger, og
rækkefølgen er (fra bedst til dårligst):
\be
21, 31,
66, 55, 44, 33, 22, 11,
65, 64, 63, 62, 61,
54, 53, 52, 51,
43, 42, 41,
32
\label{udfaldsrum}
\ee Alle dobbelt-slagene har sandsynligheden $\frac{1}{36}$ for at optræde,
mens de øvrige slag optræde med sandsynligheden $\frac{2}{36}$. Dermed bliver
medianen af slagene et sted mellem $61$ og $62$.

\section{Matematisk løsning}
I dette dokument betragter vi situationen med netop to spillere, som begge
gør deres bedste for at finde.

\subsection{Betegnelser}
Vi indfører følgende betegnelser:
$$
M : \mbox{den sidste melding}
$$

$$
T : \mbox{terningernes aktuelle visning}
$$
Således er $M$ og $T$ begge stokastiske variabler med
udfaldsrummet~(\ref{udfaldsrum}).

Dernæst får vi brug for følgende sandsynligheder:
$$
P(T=t| M=m)
$$
Dette er sandsynligheden for, at de skjulte terninger viser $t$,
givet at den sidste melding er $m$.
$$
P(T=t) : \mbox{sandsynligheden for, at terningerne viser $t$.}
$$
Til at begynde med antager vi, at en melding ikke giver nogen information om
terningnerne. Derfor antager vi, at
\be
P(T=t) = P(T=t | M=m), \quad \mbox{for alle $t$ og $m$.}
\label{uafhaengighed}
\ee

Vi indfører følgende sandsynligheder for gevinst ($m$ er den forrige spillers melding):
$$
P_L(m) : \mbox{sandsynligheden for at vinde, givet at spilleren vælger at
løfte.}
$$
$$
P_R(m) : \mbox{sandsynligheden for at vinde, givet at spilleren vælger at
ryste.}
$$
$$
P(m) : \mbox{sandsynligheden for at finde.}
$$

Til sidst får vi brug for at beskrive vores strategi.  Først har vi:
$$
R(m) : \mbox{sandsynligheden for, at vi løfter.}
$$
Vi har så umiddelbart, at der må gælde
\be
P(m) = R(m) P_R(m) + (1-R(m)) P_L(m)
\label{samlet}
\ee
fordi i enhver situation er der kun to mulige handlinger: at løfte eller at ryste.
Dernæst har vi
$$
P_M(M=m | T=t \& S=s)
$$
som er sandsynligheden for, at vi afgiver meldingen $m$, givet at terningerne
nu viser $t$ og den sidste melding var $s$.
Reglerne i spiller gør, at
$$
P_M(m|t,s) = 0 \quad {\rm for} \quad m < s
$$
Med andre ord, man må ikke afgive en melding, der er lavere end den sidste.

\subsection{Foreløbige resultater -- sidste melding er $21$}
Nu vil vi først bestemme værdien af $P_L(m=21)$: Dette svarer til situationen,
at den forrige spiller har rystet terningerne og meldt $21$. Vi vælger nu at
løfte, og hvad er så sandsynligheden for, at vi vinder?
\bas
 && P_L(m=21) \\
&=& 1 - P(t=21|m=21) , \quad \mbox{alt undtagen 21 vinder}\\
&=& 1 - P(t=21) \\
&=& \frac{34}{36} \\
&=& 0,9444
\eas
Her har vi benyttet ligning~(\ref{uafhaengighed}), hvor vi altså antager,
at den forrige spillers melding ikke giver nogen information om, hvad
terningerns visning er, og at sidstnævnte dermed er helt tilfældige.

Nu vil vi bestemme værdien af $P_R(m=21)$, dvs vi vælger at ryste terningerne
oglave en ny melding. Denne nye melding kan kun være $21$,
fordi
$$
P_M(m|t, s=21) = 0 \quad {\rm hvis} \quad m < 21
$$
og
$$
P_M(21|t, s=21) = 1
$$
Derfor har vi umiddelbart:
$$
P_R(21) = 1 - P(21)
$$
fordi vi antager, at modstanderen benytter samme strategi som os selv.

Vi benytter nu ligning~(\ref{samlet}) og omskriver:
\bas
 && P(21) \\
&=& R(21) P_R(21) + (1-R(21)) P_L(21) \\ 
&=& R(21) (1-P(21)) + (1-R(21)) P_L(21)
\eas
Vi ønsker at finde den strategi $R(21)$, som ender med at give den største
sandsynlighed for at vinde, dvs maksimere $P(21)$.

I det følgende dropper vi argumentet $21$, for at gøre notationen enklere.
Så har vi:
$$
P = R(1-P) + (1-R)P_L
$$
Vi isolerer $P$ og får
$$
P = \frac{R+(1-R)P_L}{1+R},
$$
som også kan skrives som
$$
P = 1 - P_L + \frac{2P_L-1}{1+R}
$$
Da $2P_L>1$ så ser vi, at dette er en aftagende funktion af $R$, og maksimum
opnås derfor ved $R(21) = 0$. Så hvis den sidste spiller har meldt $21$, så er
det bedst at løfte hver gang.

Vi har således fundet frem til, at
\bas
R(21) &=& 0 \\
P(21) &=& 0,9444
\eas

\subsection{Foreløbige resultater -- sidste melding er $31$}
Nu vil vi så bestemme værdien af $P_L(m=31)$, altså sandsynligheden for at
vinde, givet at vi løfter og den sidste spiller har meldt $31$.
Vi har således:
\bas
 && P_L(m=31) \\
&=& 1 - P(t \ge 31 | m=31) \\
&=& 1 - P(t \ge 31) \\
&=& \frac{32}{36} \\
&=& 0,8888
\eas

Nu vil vi så bestemme $P_R(m=31)$, dvs sandsynligheden for at vinde, givet at 
vi ryster terningnerne, og sidste spiller har meldt $31$. Vi kan nu vælge mellem
at melde $21$ og $31$. Sandsynlighederne for disse betegnes med henholdsvis $M_{21}$
og $M_{31}$. Der må derfor gælde, at $M_{31} = 1-M_{21}$.

Vi har umiddelbart, at
$$
P_R(31) = 1 - M_{21} P(21) - M_{31} P(31)
$$
Ligning~(\ref{samlet}) giver igen
$$
P(31) = R(31) P_R(31) + \big(1-R(31)\big) P_L(31)
$$
Heri indsættes ovenstående udtryk for $P_R(31)$:
$$
P(31) = R(31) \big(1 - M_{21} P(21) - M_{31} P(31)\big) + \big(1-R(31)\big) P_L(31)
$$

Nu indsætter vi $M_{21} = 1-M_{31}$ og $P(21) = 0,9444$ og $P_L(31) = 0,8888$. Dernæst
dropper vi argumentet $31$ og så kan ovenstående ligning skrives som:
$$
P = R\big(1-(1-M)0,9444-MP\big) + \big(1-R\big)0,8888
$$
I denne ligning isoleres $P$ og det giver
$$
P = 0,9444 - \frac{0,8333 R + 0,0555}{1+RM}
$$

Vi ønsker nu at maksimere værdien for $P$ givet at $0\le M \le1$ og $0\le R\le 1$.
For enhver værdi at $R$ bliver $P$ maksimal for $M=1$. Vi indsætter derfor $M=1$ i udtrykket for
$P$ og omskriver til:
$$
P = 0,1111 + \frac{0,777}{1+R}
$$
Denne værdi er størst, når $R$ er mindst, dvs $R=0$.

Vi har således fundet frem til følgende optimale værdier:
\bas
M_{31} &=& 1 \\
M_{21} &=& 0 \\
R(31) &=& 0 \\
P(31) &=& 0,8888
\eas
Det vil sige, at det optimale svar på meldingen $31$ er at løfte uanset hvad.

Bemærk, i denne analyse bruger vi slet ikke værdien af terningerne. 
Med andre ord, værdien af $M_{31}$ kan godt afhænge af den faktiske værdi af terningerne.



\end{document}


